\documentclass[11pt]{article}
% \documentclass[preview,border=24pt]{standalone}
\usepackage[T2A]{fontenc}
\usepackage[utf8x]{inputenc}
\usepackage[english, russian]{babel}
\usepackage{indentfirst}
\usepackage{array}
\usepackage{fixltx2e}
\usepackage{mathtools}
\usepackage{amssymb}
\usepackage{graphicx}
\usepackage{colortbl}
\usepackage{booktabs}
\graphicspath{ {./} }

\title{\textbf{Треугольник Паскаля как группа}}
\date{}
\begin{document}

\maketitle
Треугольник Паскаля можно представить как группу, где каждая строка - это элемент группы.
\\

\begin{tabular}{>{$ }l<{$ \hspace{12pt}}*{16}{c}}
g_0 &&&&&&&&1&&&&&&\\
g_1 &&&&&&&1&&1&&&&&\\
g_2 &&&&&&1&&2&&1&&&&\\
g_3 &&&&&1&&3&&3&&1&&&\\
g_4 &&&&1&&4&&6&&4&&1&&\\
g_5 &&&1&&5&&10&&10&&5&&1&\\
g_6 &&1&&6&&15&&20&&15&&6&&1\\
g_7 &1&&7&&21&&35&&35&&21&&7&&1\\
... &&&&&&&&...&&&&&&&&
\end{tabular}
\\

Добавим операцию '+', со свойством $g\textsubscript{n}+g\textsubscript{m} = g\textsubscript{n+m}$.
\\

Порядок чисел в элементе важен для операции, поэтому представим их как кортежи:
\[ g_0 = \{1\}, \indent g_1 = \{1, 1\}, \indent g_2 = \{1, 2, 1\}, \indent ... \]

Определим операцию как умножение столбиком без переноса разрядов.

Например,
\[g_1 + g_2 =
\begin{tabular}{ccccc}
  &  & 1 & 2 & 1 \\
(+) &  &   & 1 & 1 \\
\hline
  &   & 1 & 2 & 1 \\
+ & 1 & 2 & 1 \\
\hline
  & 1 & 3 & 3 & 1
\end{tabular}
= g_3
\]

или

\[g_4 + g_2 =
\begin{tabular}{cccccccc}
  &  &  & 1 & 4 & 6 & 4 & 1 \\
(+) &  &  &   &   & 1 & 2 & 1 \\
\hline
  &   &   & 1 & 4  & 6 & 4 & 1 \\
+ &   & 2 & 8 & 12 & 8 & 2 \\
  & 1 & 4 & 6 & 4  & 1 \\
\hline
  & 1 & 6 & 15 & 20 & 15 & 6 & 1
\end{tabular}
= g_6
\]

Нейтральным элементом в группе является $g\textsubscript{0}$, т.к. для любого $n$ выполняется $g\textsubscript{n} + g\textsubscript{0} = g\textsubscript{n}$.

Треугольник Паскаля может быть определен для отрицательных строк\footnote{Конкретная математика. 1998. Грэхем, Кнут, Поташник. Стр. 189, таблица 189}, которые будут обратными элементами группы. Элементы треугольника - биномиальные коэффициенты, который можно определеить при ${n\choose k}$, где $k\in \mathbb{N}$, а $n\in \mathbb{Z}$, т.е. $n$ может быть отрицательным.
Пример, отрицательных строк треугольника:

\begin{tabular}{>{$ }l<{$ \hspace{12pt}}*{16}{c}}
...    &&&&...&&&&\\
g_{-3} &1&-3&6&-10&15&-21&...\\
g_{-2} &1&-2&3&-4&5&-6&...\\
g_{-1} &1&-1&1&-1&1&-1&...\\
g_0    &1\\
\end{tabular}
\\

Примеры операции с отрицательными элементами:
\[g_{-1} + g_1 =
\begin{tabular}{cccccccc}
  & 1 & -1 & 1 & -1 & 1 & -1 & ...\\
(+) &   &    &   &    & 1 & 1 \\
\hline
  & 1 & -1  &  1 & -1 &  1 & -1 & ...\\
+ &   &  1  & -1 &  1 & -1 &  1 & ...\\
\hline
& 1 & 0 & 0 & 0 & 0 & 0 & ...\\
\end{tabular}
= g_0
\]

$ $

\[g_{-2} + g_1 =
\begin{tabular}{cccccccc}
  & 1 & -2 & 3 & -4 & 5 & -6 & ...\\
(+) &   &    &   &    & 1 & 1 \\
\hline
  & 1 & -2  &  3 & -4 &  5 & -6 & ...\\
+ &   &  1  & -2 &  3 & -4 &  5 & ...\\
\hline
& 1 & -1 & 1 & -1 & 1 & -1 & ...\\
\end{tabular}
= g_{-1}
\]

В силу наличия свойств ассоциативности и коммутативности у целых чисел, получаем, что данное множество с операцией также обладает этими свойствами. В нем есть нейтральный элемент и для каждого элемента есть обратный, отсюда следует, что оно является группой (обозначим $G$). Также эта группа изоморфна группе целых чисел по сложению, $(G, +) \cong (\mathbb{Z}, +)$.
\end{document}
